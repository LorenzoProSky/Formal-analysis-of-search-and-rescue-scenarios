
\section{Conclusions}
Our analysis of search-and-rescue scenarios using UPPAAL has yielded several important insights:

\begin{enumerate}
	\item The effectiveness of rescue strategies varies significantly across different emergency situations, underscoring the necessity for adaptable approaches.
	
	\item Time constraints emerge as a critical factor in rescue operations. Extended timeframes generally correlate with higher success rates, particularly when attempting to rescue larger groups of civilians.
	
	\item The stochastic analysis reveals a stark contrast between single-civilian and large-scale rescues. While individual rescues demonstrate high reliability, operations involving a larger percentage of civilians (e.g., 25\%) are considerably more time-sensitive and less predictable.
	
	\item Notably, none of the modelled systems achieved a consistent 100\% rescue rate, highlighting the inherent complexities and uncertainties in emergency response scenarios.
	
	\item The model effectively illustrates the intricate interplay of various factors, including environmental conditions, resource availability (responders and drones), and civilian vulnerability levels.
	
	\item Potential areas for improvement include optimising resource allocation, enhancing the strategic capabilities of drones and responders, and developing more versatile systems capable of adapting to diverse emergency contexts.
\end{enumerate}

In conclusion, this study demonstrates the significant value of formal modelling techniques in analysing and enhancing emergency response systems. While providing valuable insights into current capabilities, it also illuminates promising avenues for future research and development in this critical field.