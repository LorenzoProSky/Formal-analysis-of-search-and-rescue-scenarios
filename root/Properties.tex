
\section{Properties}

\subsection{Scenarios}

We have designed three scenarios to evaluate our emergency response system: Urban High-Rise Fire, Large-Scale Natural Disaster, and Industrial Complex Emergency. The following table summarizes their key parameters:
\begin{table}[h]
	\centering
	\begin{tabular}{|l|c|c|c|}
		\hline
		\textbf{Parameter} & \textbf{Scenario 1} & \textbf{Scenario 2} & \textbf{Scenario 3} \\
		\hline
		Environment & Urban High-Rise & Natural Disaster & Industrial Complex \\
		Map Size & 8x10 & 20x14 & 18x12 \\
		Drones (visibility) & 2 ($n_v = 2$) & 2 ($n_v = 3$) & 4 ($n_v = 1$) \\
		First Responders (rescue time) & 2 ($t_{fr} = 2$-3) & 5 ($t_{fr} = 8$) & 5 ($t_{fr} = 8$) \\
		Civilians (vulnerability, & 3 ($t_v = 2$-11, & 11 ($t_v = 15$-25, & 10 ($t_v = 15$-25, \\
		zero-responder rescue time) & $t_{zr} = 1$-2) & $t_{zr} = 6$) & $t_{zr} = 6$) \\
		\hline
	\end{tabular}
	\caption{Summary of Emergency Response Scenarios}
	\label{tab:scenarios}
\end{table}

These scenarios assess system performance in different conditions, such as dense urban vs. widespread disaster areas, limited vs. advanced drone capabilities, and varying civilian vulnerability.
\subsubsection{Scenario 1: Urban High-Rise Fire}
Tests efficiency in a compact, high-stakes environment. The compact map simulates a dense urban environment with advanced drones with for example thermal imaging capabilities. Limited responders reflect initial teams, while varied civilian vulnerabilities test prioritization in time sensitive situations.
\subsubsection{Scenario 2: Large-Scale Natural Disaster}
Tests handling of complex, large-scale situations with limited but advanced resources. The large map represents a widespread disaster area. Advanced drones test efficient resource allocation over a vast area. Similar high rescue times reflect environmental challenges, while high civilian vulnerability simulates a population caught unprepared.
\subsubsection{Scenario 3: Industrial Complex Emergency}
Tests operations in a hazardous environment with low-visibility drones. The medium-sized map simulates varied industrial terrain. Low-visibility drones reflect operational challenges in potentially hazardous conditions. High rescue times account for the complex environment, while varied civilian vulnerabilities represent different risk levels across the industrial complex.

\subsection{Verification Results (Non-Stochastic)}

This section discusses the results of two queries on three systems to assess the efficiency and reliability of civilian rescue operations under various constraints. The key variables were $N\%$ (percentage of civilians to be rescued) and $T_{scs}$ (time constraint for rescue).

The queries were:


\begin{quote}
	\raggedright
	\begin{itemize}
		\item \texttt{E<> n\_safe >= n\_percentage \&\& global\_time < t\_scs \&\& initializer.Initialized}
		\item \texttt{A<> n\_safe >= n\_percentage \&\& global\_time < t\_scs \&\& initializer.Initialized}
	\end{itemize}
\end{quote}

These queries evaluate the possibility (\texttt{E<>}) and inevitability (\texttt{A<>}) of rescuing a certain percentage of civilians within a time frame.

For the tables 'V' indicates a successful query, while '$\infty$' means that the query was running for a long time meaning the result could be successfull or not (but is very likely it will eventually turn not sucessfull). 

\subsubsection{System 1 Results}

The second query, checking if all paths lead to rescuing at least $N\%$ of civilians within $T_{scs}$, always returned false, indicating no guaranteed rescue within the time constraints. The first query, checking if at least one path allows rescuing $N\%$ of civilians within $T_{scs}$, gave more varied results, summarized below:

\begin{table}[H]
	\centering
	\small
	\begin{tabular}{|c|c|c|c|c|}
		\hline
		$T_{scs}\backslash N\%$ & 25\% & 50\% & 75\% & 100\% \\ \hline
		2                       & $\infty$   & $\infty$   & $\infty$   & $\infty$    \\ \hline
		3                       & V   & $\infty$   & $\infty$   & $\infty$    \\ \hline
		4                       & V   & V   & V   & $\infty$    \\ \hline
		5                       & V   & V   & V   & V    \\ \hline
	\end{tabular}
	\caption{System 1 Query 1 Results}
	\label{tab:system1}
	\normalsize
\end{table}

These results show that significant rescues are possible given enough time, highlighting the importance of quick response and efficient resource allocation early in rescue operations.

\subsubsection{System 2 Results}
For the second system, our analysis revealed distinct patterns in the rescue operation's dynamics. 
The outcomes for the first query are summarized in the following table:

\begin{table}[H]
	\centering
	\small
	\begin{tabular}{|c|c|c|c|c|}
		\hline
		$T_{scs}\backslash N\%$ & 25\% & 50\% & 75\% & 100\% \\ \hline
		1                     & V   & $\infty$  & $\infty$   &$\infty$    \\ \hline
		2                       & V   & $\infty$  & $\infty$   &$\infty$    \\ \hline
		5                      & V   & $\infty$  & $\infty$   &$\infty$    \\ \hline
		10                      & V   & $\infty$  & $\infty$   &$\infty$    \\ \hline
		100                       & V   & V  & $\infty$   &$\infty$    \\ \hline
	\end{tabular}
	\caption{System 2 Query 1 Results}
	\label{tab:system1}
	\normalsize
\end{table}

For the first query, the results show that it returns true if and only if the number of people to be saved is 1, regardless of the time constraint.

These findings suggest that System 2 has a more constrained rescue capability compared to System 1. It provides a consistent level of performance for small-scale rescues but faces significant challenges in scaling up to larger percentages of the civilian population.


\subsubsection{System 3 Results}

System 3 exhibits characteristics similar to System 2, with the difference that query 2 returns always false.

Key observations for System 3:
\begin{enumerate}
	\item The system maintains the possibility of saving up to 25\% of civilians across all time frames.
	\item As with System 2, increasing the time constraint does not improve rescue potential for larger groups.
	\item The false result for Query 2 suggests that even saving a single civilian is not guaranteed under all circumstances.
	\item The system shows no successful scenarios for rescuing 50\% or more of the civilians.
\end{enumerate}

System 3 presents a more challenging rescue environment compared to Systems 1 and 2. While it retains the potential for medium-scale rescues, it lacks any guaranteed successful outcomes, indicating a higher level of uncertainty in rescue operations.

\subsection{Verification Results (Stochastic)}
In addition to the deterministic queries, we decided to conduct the stochastic analysis on the second scenario using the following query:

\begin{quote}
	\texttt{Pr[<=t\_scs](<> n\_safe >= n\_percentage \&\& initializer.Initialized)}
\end{quote}

Key findings include:

\begin{itemize}
	\item \textbf{Guaranteed Minimal Rescue:} (trivial) When the target was to save just one civilian, the query returned a 100\% pass rate, regardless of the time constraint.
	\item \textbf{Time-Dependent Success Rate:} For saving 25\% of civilians:
	\begin{itemize}
		\item With $t_{scs} = 10$, the success probability was 30\%.
		\item With $t_{scs} = 2$, the success probability decreased to 17\%.
	\end{itemize}
\end{itemize}

These results indicate that while single-civilian rescue is highly reliable, the success rate for larger-scale rescues (25\% of civilians) is significantly time-dependent, with longer time frames improving the chances of success.